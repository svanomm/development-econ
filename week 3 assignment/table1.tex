\begin{table}[H]
\caption*{
{\large Productivity Calculations: Ratios to U.S. Values}
} 

\fontsize{12.0pt}{14.4pt}\selectfont

\begin{tabular*}{\linewidth}{@{\extracolsep{\fill}}lrrrr}
\toprule
Country & $Y/L$ & $(K/Y)^{\alpha/(1-\alpha)}$ & $H/L$ & A \\ 
\midrule\addlinespace[2.5pt]
U.S.A. & 1.000 & 1.000 & 1.000 & 1.000 \\
Canada & 0.941 & 1.002 & 0.908 & 1.034 \\
Italy & 0.834 & 1.063 & 0.650 & 1.207 \\
Germany, West & 0.818 & 1.118 & 0.802 & 0.912 \\
France & 0.818 & 1.091 & 0.666 & 1.126 \\
U.K. & 0.727 & 0.891 & 0.808 & 1.011 \\
Hong Kong & 0.608 & 0.741 & 0.735 & 1.115 \\
Singapore & 0.606 & 1.031 & 0.545 & 1.078 \\
Japan & 0.587 & 1.119 & 0.797 & 0.658 \\
Mexico & 0.433 & 0.868 & 0.538 & 0.926 \\
Argentina & 0.418 & 0.953 & 0.676 & 0.648 \\
U.S.S.R. & 0.417 & 1.231 & 0.724 & 0.468 \\
India & 0.086 & 0.709 & 0.454 & 0.267 \\
China & 0.060 & 0.891 & 0.632 & 0.106 \\
Kenya & 0.056 & 0.747 & 0.457 & 0.165 \\
Zaire & 0.033 & 0.499 & 0.408 & 0.160 \\
\bottomrule
\end{tabular*}
\begin{minipage}{\linewidth}
The elements of this table are the empirical counterparts to the components of equation (3), all measured as ratios to the U. S. values. That is, the first column of data is the product of the other three columns.\\
\end{minipage}
\end{table}
